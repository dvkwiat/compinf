
\begin{figure}[htbp]
\begin{tikzpicture}
\begin{axis}[no markers, legend style={at={(1.1,0.5)}, anchor=west, legend columns=1}, xlabel={Number of firms}, ylabel={Price informativeness}, y label style={at={(axis description cs:-0.2,0.5)}, anchor=south}, legend cell align={left}, xtick={1, 2, 3}, xticklabels={1, 2, 3}, ytick={0.5, 1}, yticklabels={50\%, 100\%}, ultra thick, xmin=0.75, xmax=3.25, ymin=0.4, ymax=1.1]

\addplot [color=black, solid] table [x=pool_n, y=pool_inf, col sep=comma, unbounded coords=jump] {../output/Theory Section/informativeness.data};
\addlegendentry{Pooling at $p_L$}
\addplot [color=red, dotted] table [x=lmix_n, y=lmix_inf, col sep=comma, unbounded coords=jump] {../output/Theory Section/informativeness.data};
\addlegendentry{Low-type mixes}
\addplot [color=blue, dashed] table [x=hmix_n, y=hmix_inf, col sep=comma, unbounded coords=jump] {../output/Theory Section/informativeness.data};
\addlegendentry{High-type mixes}

\end{axis}
\end{tikzpicture}
\caption{Price informativeness as competition increases}
\label{informativeness}
\begin{minipage}{0.8\textwidth}
\footnotesize
\textit{Note:} This figure is created with the parameters used in the experiment. There always exists an $N \in \mathbb{N}$ such that informativeness has fully converged for all $n \geq N$; this $N$ is always greater than 1, but can be made arbitrarily large by the choice of parameters.
\end{minipage}
\end{figure}
